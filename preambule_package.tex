\usepackage[utf8]{inputenc} % encodage entree
\usepackage[square,numbers]{natbib} % Bibliography
\usepackage[main = french,english,greek]{babel} % français
\usepackage[T1]{fontenc} % encodage sortie
\usepackage{amsmath} % polices
\usepackage{amssymb} % symboles
\usepackage{amsthm} % th, prop etc.
\usepackage{xparse}
\usepackage{geometry} % correction footnote
\usepackage{fancyhdr} % entêtes et pied-de-page
\usepackage{appendix} % annexes
\usepackage{graphicx}  % images
\usepackage{xcolor} % couleur
\usepackage{mathrsfs} % police maths /mathscr
\usepackage{array} % tableau
\usepackage{lmodern} % police
\usepackage{slantsc} % correction police
\usepackage{fix-cm} % correction CM
\usepackage{chemfig} % formules chimiques
\usepackage[version=4]{mhchem}
\usepackage{listings} % code
\usepackage{lipsum} % lorem ipsum
\usepackage{tikz} % schem
\usepackage[RPvoltages, european, straight voltages, cuteinductors]{circuitikz} % circuits
\usepackage{tikz-3dplot} % plot 3D
\usepackage{esint} % intégrales
\usepackage{textgreek} % lettres grecques en roman
\usepackage{setspace} %  interlignes
\usepackage{lastpage} %  dernière page
\usepackage{paracol} % plusieurs colones
\usepackage[bottom]{footmisc} % correction footnote
\usepackage{lscape} % format paysage pour une page : environnement landscape (remplacer par le package pdflscape si besoin)
\usepackage[arrowdel]{physics} % Un package essentiel
\usepackage{afterpage} % pour intégrer le \clearpage dès la fin de page
\usepackage{caption}% figure...
\usepackage{subcaption}% sous-figures...
\usepackage{multirow} % case de plusieurs lignes dans un tableau
\usepackage{datetime} % date et heure de compilation
\usepackage{hyperref} % ref. url etc... en dernier !

%%%%%%%%%%%%%%%%%%%%%%%%%%%%%%%%%%%%%%%%%%%%%%%%%%%%%
%            Corrections et paramètres              %
%%%%%%%%%%%%%%%%%%%%%%%%%%%%%%%%%%%%%%%%%%%%%%%%%%%%%
\mathcode`\.="013B % virgule à la place du point en séparation décimale
\setlength\parindent{0cm} % pas d'indentation des paragraphes
\renewcommand{\oiint}{\varoiint} % intégrales

\rmfamily
\DeclareFontShape{T1}{lmr}{b}{sc}{<->ssub*cmr/bx/sc}{}
\DeclareFontShape{T1}{lmr}{bx}{sc}{<->ssub*cmr/bx/sc}{}
\DeclareFontShape{T1}{lmr}{m}{scit}{<->ssub*cmr/m/scit}{}

%%%%%%%%%%%%%%%%%%%%%%%%%%%%%%%%%%%
\bibliographystyle{siam} %siam, yii
%%%%%%%%%%%%%%%%%%%%%%%%%%%%%%%%%%%

\numberwithin{equation}{section}
\lstset{frame=single, numbers=left, showstringspaces=false, basicstyle=\footnotesize\ttfamily, breaklines=true, commentstyle=\color{ym_bleu}, keywordstyle=\color{ym_jaune}, stringstyle=\color{ym_vert}, extendedchars=true, literate={à}{{a}}1 {â}{{a}}1 {é}{{e}}1 {è}{{e}}1 {ê}{{e}}1 {ë}{{e}}1 {ï}{{i}}1 {û}{u}1 {ù}{u}1 {î}{i}1 {À}{{A}}1 {Â}{{A}}1 {É}{{E}}1 {È}{{E}}1 {Ê}{{E}}1 {Ë}{{E}}1 {Ï}{{I}}1 {Û}{U}1 {Ù}{U}1 {Î}{I}1}

\usetikzlibrary{babel}
\usetikzlibrary{decorations.pathmorphing, decorations.markings}
\tikzset{gluon/.style={thick, decorate, decoration={coil, segment length=4pt}}}
\tikzset{boson/.style={thick, decorate, decoration={coil,aspect=0}}}
\tikzset{higgs/.style={thick, dashed}}
\tikzset{fermion/.style={thick, postaction={decorate}, decoration={markings,mark=at position 0.55 with {\arrow{latex}}}}}
\tikzset{antifermion/.style={thick, postaction={decorate}, decoration={markings,mark=at position 0.55 with {\arrow{latex reversed}}}}}

% \hypersetup{unicode=false}
%%%%%%%%%%%%%%%%%%%%%%%%%%%%%%%%%%%%%%%%%%%%%%%%%%%%%
%                     Couleurs                      %
%%%%%%%%%%%%%%%%%%%%%%%%%%%%%%%%%%%%%%%%%%%%%%%%%%%%%
\definecolor{ym_noir}{HTML}{121212}
\definecolor{ym_blanc}{HTML}{dddddd}
\definecolor{ym_bleu}{HTML}{003763}
\definecolor{ym_bleuc}{HTML}{3898ff}
\definecolor{ym_jaune}{HTML}{DAA21C}
\definecolor{ym_vert}{HTML}{10AA10}
\definecolor{ym_violet}{HTML}{AA10AA}

\colorlet{fg}{black}
\colorlet{bg}{white}

%%%%%%%%%%%%%%%%%%%%%%%%%%%%%%%%%%%%%%%%%%%%%%%%%%%%%
%                     Commandes                     %
%%%%%%%%%%%%%%%%%%%%%%%%%%%%%%%%%%%%%%%%%%%%%%%%%%%%%
% renewmathoperator
\makeatletter
\newcommand\RedeclareMathOperator{%
  \@ifstar{\def\rmo@s{m}\rmo@redeclare}{\def\rmo@s{o}\rmo@redeclare}%
}
\newcommand\rmo@redeclare[2]{%
  \begingroup \escapechar\m@ne\xdef\@gtempa{{\string#1}}\endgroup
  \expandafter\@ifundefined\@gtempa
     {\@latex@error{\noexpand#1undefined}\@ehc}%
     \relax
  \expandafter\rmo@declmathop\rmo@s{#1}{#2}}
\newcommand\rmo@declmathop[3]{%
  \DeclareRobustCommand{#2}{\qopname\newmcodes@#1{#3}}%
}
\@onlypreamble\RedeclareMathOperator
\makeatother
% basdepage
\newcommand\basdepage[1]{%
  \begingroup
  \renewcommand\thefootnote{}\footnote{#1}%
  \addtocounter{footnote}{-1}%
  \endgroup
}


% blackboard
\newcommand{\ens}[1]{\mathbf{#1}}
\newcommand{\esp}[1]{\mathcal{#1}}
\newcommand{\fk}[1]{\mathfrak{#1}}
\newcommand{\mat}[1]{\mathcal{#1}} % equiv \esp
\newcommand{\N}{\ens{N}}
\newcommand{\Z}{\ens{Z}}
\newcommand{\D}{\ens{D}}
\newcommand{\Q}{\ens{Q}}
\newcommand{\R}{\ens{R}}
\newcommand{\C}{\ens{C}}
\newcommand{\K}{\ens{K}}
\newcommand{\etoile}{^{\displaystyle \ast}}
\newcommand{\adj}{^{\dagger}}
\newcommand{\std}{\text{°}}
\newcommand{\odeg}{\text{°}}
\newcommand{\fonc}[2]{\left\langle {#1},\,{#2}\right\rangle}
\newcommand{\sfonc}[2]{\langle {#1},\,{#2}\rangle}

% opérateurs
\DeclareMathOperator{\pgcd}{pgcd}
%\DeclareMathOperator{\division}{\div}
\DeclareMathOperator{\Vect}{Vect}
\DeclareMathOperator{\Ker}{Ker}
\DeclareMathOperator{\Image}{Im}
\DeclareMathOperator{\Sp}{Sp}
%\DeclareMathOperator{\grad}{\vect{\nabla}} % intégré à physics
%\RedeclareMathOperator{\div}{\vect{\nabla}\cdot} % intégré à physics
\newcommand{\vect}[1]{\va*{#1}}
\newcommand{\rot}{\curl}
\DeclareMathOperator{\lap}{\laplacian}
\DeclareMathOperator{\vlap}{\vect{\nabla}^2}
\DeclareMathOperator{\ham}{\mathcal H}
\DeclareMathOperator{\lag}{\mathcal L}
\DeclareMathOperator{\mom}{\mathcal M}
\DeclareMathOperator{\lang}{\mathscr L}
\DeclareMathOperator{\card}{card}
\DeclareMathOperator{\p}{\mathbb P}
\DeclareMathOperator{\arctandeux}{arctan2}
%\DeclareMathOperator{\cotan}{cotan} % intégré à physics
\DeclareMathOperator{\dal}{\square}
\DeclareMathOperator{\TF}{TF}
\DeclareMathOperator{\FT}{FT}
\DeclareMathOperator{\VP}{VP}

\newcommand{\unite}[1]{\mathrm{\ #1}}
\newcommand{\e}{\mathrm{e}}
\newcommand{\im}{\mathrm{i}}
\newcommand{\jm}{\mathrm{j}}
\newcommand{\E}[1]{\times 10^{#1}}
\newcommand{\cqfd}{\hspace*{0pt} \hfill $\square$}
\newcommand{\NA}{\mathcal{N}_\mathrm{A}}
\newcommand{\kB}{k_\mathrm{B}}
\newcommand{\G}{\mathcal{G}}
\newcommand{\ul}{\underline}
\newcommand{\ol}{\overline}
\newcommand{\mathmu}{\text{\normalfont{\textmu}}}
\newcommand{\mathOmega}{\text{\normalfont{\textOmega}}}
\newcommand{\mathalpha}{\text{\normalfont{\textalpha}}}
\newcommand{\mathbeta}{\text{\normalfont{\textbeta}}}
\newcommand{\mathgamma}{\text{\normalfont{\textgamma}}}
\newcommand{\mathtau}{\text{\normalfont{\texttau}}}
\newcommand{\mathnu}{\text{\normalfont{\textnu}}}
\newcommand{\act}{\mathcal S}
\newcommand{\eqdef}{\overset{\mathrm{def}}{=}}
\newcommand{\cste}{\mathrm{cste}}
\newcommand{\cst}{\mathrm{cst}}
\newcommand{\moy}[1]{\left\langle{#1}\right\rangle}
\newcommand{\smoy}[1]{\langle{#1}\rangle}
\newcommand{\ope}[1]{\hat{#1}}
\newcommand{\opevect}[1]{\hat{\vb{#1}}}
\newcommand{\sol}{\odot}
\newcommand{\ter}{\oplus}
\newcommand{\uma}{\mathrm{u}}
\DeclareDocumentCommand\atom{m g g g}{
  \IfNoValueTF{#2}{\ce{#1}}{
    \IfNoValueTF{#3}{\ce{^{#2}{#1}}}{
      \IfNoValueTF{#4}{\ce{^{#2}_{#3}{#1}}}{
        \ce{^{#2}_{#3}{#1}_{#4}}
      }
    }
  }
}
\DeclareDocumentCommand\underbraceset{m g}{
  \IfNoValueTF{#2}{\underbrace{#1}}{
    \underset{#1}{\underbrace{#2}}
  }
}
% vecteurs unitaires
\newcommand{\ux}{\vu{u}_x}
\newcommand{\uy}{\vu{u}_y}
\newcommand{\uz}{\vu{u}_z}
\newcommand{\ur}{\vu{u}_r}
\newcommand{\un}{\vu{u}_n}
\newcommand{\ut}{\vu{u}_t}
\newcommand{\urho}{\vu{u}_\rho}
\newcommand{\utheta}{\vu{u}_\theta}
\newcommand{\uvartheta}{\vu{u}_\vartheta}
\newcommand{\uphi}{\vu{u}_\phi}
\newcommand{\uvarphi}{\vu{u}_\varphi}

\newcommand{\ex}{\vu{e}_x}
\newcommand{\ey}{\vu{e}_y}
\newcommand{\ez}{\vu{e}_z}
\newcommand{\er}{\vu{e}_r}
\newcommand{\en}{\vu{e}_n}
\newcommand{\et}{\vu{e}_t}
\newcommand{\erho}{\vu{e}_\rho}
\newcommand{\etheta}{\vu{e}_\theta}
\newcommand{\evartheta}{\vu{e}_\vartheta}
\newcommand{\ephi}{\vu{e}_\phi}
\newcommand{\evarphi}{\vu{e}_\varphi}

% d, déron, etc.
\newcommand{\dx}{\deh{x}}
\newcommand{\dy}{\deh{y}}
\newcommand{\dz}{\deh{z}}
\newcommand{\dt}{\deh{t}}
\newcommand{\dr}{\deh{r}}
\newcommand{\drho}{\deh{\rho}}
\newcommand{\dtheta}{\deh{\theta}}
\newcommand{\dvartheta}{\deh{\vartheta}}
\newcommand{\dphi}{\deh{\phi}}
\newcommand{\dvarphi}{\deh{\varphi}}
\newcommand{\deh}{\dd}
\newcommand{\deron}{\partial}
\newcommand{\ddr}{\dd[3]{\vect r}}
\newcommand{\ddp}{\dd[3]{\vect p}}

%\newcommand{\sderon}{\text{{\scriptsize $\partial$}}} % en attendant de trouver mieux solution…
\newcommand{\Deh}{\mathrm{D}}
\newcommand{\dehlta}{\var} % en fait non utilisé...
\newcommand{\ie}{\textit{i.e.}}
\newcommand{\etal}{\textit{et al}}

\newenvironment{question}[1]{\textbf{Q\,#1.}}{}
\newenvironment{systeme}{\left\{\begin{array}{ll}}{\end{array}\right.} % Attention, l'accolade est bien fermée ici !
